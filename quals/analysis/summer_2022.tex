\documentclass{exam}
\usepackage{amsmath}
\usepackage{amssymb}
\usepackage{amsthm}
\usepackage{graphicx}
\usepackage{float}
\usepackage{mathtools}
\usepackage[T1]{fontenc}

\renewcommand{\questionlabel}{\textbf{R-1.A.}}
% \renewcommand{\questionlabel}{\textbf{R-1}.~\thequestion.}

\begin{document}
\begin{center}

\makebox[\textwidth]{\textsc{Analysis PhD Qualifying Exam}}
\makebox[\textwidth]{\textsc{August 15, 2022}}
\makebox[\textwidth]{\textsc{Ionuţ Chifan and Raúl Curto}}
\vspace{1em}

\section*{\textsc{Part I: Real Analysis}}
\end{center}
\vspace*{1cm}
Choose either problem \textbf{R-1A} or \textbf{R-1B} below and solve it.\\

\begin{questions}
    \question[10]
        Let $E\coloneq (0, +\infty),\ g\in L^1(E)$, and define $f: E\to \mathbb{R}$ by 
        \[f(x)\coloneq\int_{(0,x)}g \ (x\in E).\]
        \begin{parts}
            \part
                Prove that $f$ is absolutely continuous on $E$.
            \part
                Prove that $f$ may fail to be Lipschitz on $E$, that is, find $g\in L^1(E)$ such that the associated $f$ is not Lipschitz on $E$.
            \part
                Prove that when $g \in L^\infty(E)$, then $f$ is Lipschitz.
            \part
                For $g\in L^\infty$, find the best Lipschitz constant for $f$.\\
        \end{parts}

        \renewcommand{\questionlabel}{\textbf{R-1B.}}

    \question[10]
        Let $\{E_k\}_{k=1}^n$ be a finite family of measurable subsets of $[0,1]$. Assume that every $x\in [0,1]$ belongs to at least three sets in the family. 
        Prove that there exists $k = 1, \dots, n$ such that 
        \[m(E_k)\geq\frac{3}{n},\]
        where $m$ denotes Lebesgue measure.\\

        \renewcommand{\questionlabel}{\textbf{R-2.}}

    \question[10]
        Consider the sequence of real valued functions on $[0,1]$ given by
        \[f_n(x)\coloneq
        \begin{cases}
           2n & \frac{1}{2n} \leq x \leq \frac{1}{n}\\
           0 & x\in \left[0, \frac{1}{2n}\right) \cup \left(\frac{1}{n},\right]. 
        \end{cases}\]
        \begin{parts}
            \part
                Find
                \[\int_0^1 \lim_n f_n.\]
            \part
                Find 
                \[\lim_n\int_0^1 f_n.\]
            \part
                Does Fatou's Lemma apply to the sequence $\{f_n\}$? Why or why not?
            \part
                Does the Lebesgue Dominated Convergence Theorem apply to the sequence $\{f_n\}$? Why or why not?
        \end{parts}

\end{questions}


\end{document}