\documentclass{exam}
\usepackage{amsmath}
\usepackage{amssymb}
\usepackage{mathtools}


\begin{document}
\centering
\makebox[\textwidth]{\textsc{Real Analysis Final}}
\makebox[\textwidth]{\textsc{Fall 2021}}
\makebox[\textwidth]{\textsc{Raúl Curto}}
\vspace{1em}

\begin{questions}
    \question[12]
        Define:
        \begin{parts}
            \part[2]
                Uniform convergence of a sequence of functions on a set $E\subseteq \mathbb{R}$.
            \part[3]
                Variation of a function $f$ on a closed, bounded interval $[a,b]$, with respect to a partition $P$ of $[a,b]$.
            \part[2]
                Point of closure of a subset $E$ of a metric space $X$.
            \part[2]
                Equicontinuity for a colleciton $\mathcal{F}$ of real-valued functions on a metric space $X$.
            \part[3]
                State the Cantor intersection theorem.
        \end{parts}
    \question[8]
        Let $f$ be a continuous function on a closed, bounded, nondegenerate interval $[a,b]$ such that 

        (i) $f$ is of bounded variation on $[a,b]$; and

        (ii) $f$ maps sets of measure zero to sets of measure zero; that is, for $E$ a measurable subset of $[a,b]$, $m(E) = 0 \implies m(f(E)) = 0$.

        Prove that $f$ is absolutely continuous on $[a,b]$.
    \question[8]
        Let $f: \mathbb{R} \to \mathbb{R}$ be a Borel function, and define 
        \[\mu(E) \coloneq m(f^{-1}(E)) \quad (E \subseteq \mathbb{R}, E \text{ Borel}).\]
        Prove:
        \begin{parts}
            \part
                $\mu(E)\geq 0$ for all $E$ Borel;
            \part
                $\mu$ is monotone;
            \part
                $\mu$ is countably additive;
            \part
                $\mu$ is not translation invariant; e.g., find a counterexample of a Borel function $f$ and a set $E$ such that $\mu(E+1)\not = \mu(E)$, where $E+1 \coloneq \{x+1\mid x\in E\}$.
        \end{parts}
    \question[20]
    % TODO: question 4
        \begin{parts}
            \part[4]
            \part[6]
            \part[4]
            \part[6]
        \end{parts}
    \question[16]
    % TODO: question 5
        \begin{parts}
            \part[6]
            \part[10]
                \begin{subparts}
                    \subpart[2]
                    \subpart[2]
                    \subpart[3]
                    \subpart[3]
                \end{subparts}
        \end{parts}
    \question[6]
    % TODO: question 6
    \question[16]
    % TODO: question 7
        \begin{parts}
            \part[4]
            \part[6]
            \part[6]
        \end{parts}
    \question[14]
    % TODO: question 8
        \begin{parts}
            \part[4]
            \part[3]
            \part[4]
            \part[3]
        \end{parts}
\end{questions}

\end{document}