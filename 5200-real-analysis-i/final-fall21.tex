\documentclass{exam}
\usepackage{amsmath}
\usepackage{amssymb}
\usepackage{mathtools}


\begin{document}
\centering
\makebox[\textwidth]{\textsc{Real Analysis Final}}
\makebox[\textwidth]{\textsc{Fall 2021}}
\makebox[\textwidth]{\textsc{Raúl Curto}}
\vspace{1em}

\begin{questions}
    \question[12]
        Define:
        \begin{parts}
            \part[2]
                Uniform convergence of a sequence of functions on a set $E\subseteq \mathbb{R}$.
            \part[3]
                Variation of a function $f$ on a closed, bounded interval $[a,b]$, with respect to a partition $P$ of $[a,b]$.
            \part[2]
                Point of closure of a subset $E$ of a metric space $X$.
            \part[2]
                Equicontinuity for a colleciton $\mathcal{F}$ of real-valued functions on a metric space $X$.
            \part[3]
                State the Cantor intersection theorem.
        \end{parts}
    \question[8]
        Let $f$ be a continuous function on a closed, bounded, nondegenerate interval $[a,b]$ such that 

        (i) $f$ is of bounded variation on $[a,b]$; and

        (ii) $f$ maps sets of measure zero to sets of measure zero; that is, for $E$ a measurable subset of $[a,b]$, $m(E) = 0 \implies m(f(E)) = 0$.

        Prove that $f$ is absolutely continuous on $[a,b]$.
    \question[8]
        Let $f: \mathbb{R} \to \mathbb{R}$ be a Borel function, and define 
        \[\mu(E) \coloneq m(f^{-1}(E)) \quad (E \subseteq \mathbb{R}, E \text{ Borel}).\]
        Prove:
        \begin{parts}
            \part
                $\mu(E)\geq 0$ for all $E$ Borel;
            \part
                $\mu$ is monotone;
            \part
                $\mu$ is countably additive;
            \part
                $\mu$ is not translation invariant; e.g., find a counterexample of a Borel function $f$ and a set $E$ such that $\mu(E+1)\not = \mu(E)$, where $E+1 \coloneq \{x+1\mid x\in E\}$.
        \end{parts}
    \question[20]
        Determine if each of the following statements is true or false. If true, provide a proof; if false, 
        provide a counterexample or show in some fashion why the statement is false. In either case, you are free 
        to cite the textbook, and provide a rationale along the lines of ``...by a proposition in Section a.b of Royden-Fitzpatrick.''
        \begin{parts}
            \part[4]
                True or false? Let $\mathcal{F}$ be a collection of measurable functions on $\mathbb{R}$, and let 
                \[g\coloneq \sup_{f\in\mathcal{F}}f.\]
                Then $g$ is measurable on $\mathbb{R}$.
            \part[6]
                True or false? Consider the normed linear space $X$ of Riemann integrable functions on $[0,1]$, with the norm $\|f\|_R\coloneq (R)\int_0^1|f(x)|dx$. Then $X$ is a Banach space. 
                (Hint: the sequence $\{f_n\}_{n=1}^\infty$ of measurable functions given by 
                \begin{equation*}
                    f_n(x)\coloneq
                    \begin{cases}
                        1 & x=\frac{i}{k}, \text{ if } 1\leq k\leq n \text{ and } 0\leq i\leq k\\
                        0 & \text{otherwise.}
                    \end{cases}
                \end{equation*}
                The sequence $\{f_n\}$ is increasing and $f_n \to \chi_\mathbb{Q}$ as $n\to\infty$.)
            \part[4]
                True or false? Recall that $\ell^\infty$ is the Banach space of real bounded sequences, equipped with the supremum norm. The space $\ell^\infty$ is separable. 
                (Hint: $2^\mathbb{N}$ is not countable.)
            \part[6]
                True or false? Let $g$ be strictly increasing and absolutely continuous on a closed, bounded, nondegenerate interval $[a,b]$, and let $\mathcal{O}$ be an open subset of $(a,b)$. Then 
                \[m(g(\mathcal{O})) = \int_\mathcal{O}g'.\]
        \end{parts}
    \question[16]
    % TODO: question 5
        \begin{parts}
            \part[6]
                A Hilbert space $\mathcal{H}$ may be characterized as a Banach space in which the so-called parallelogram law holds; that is, for each pair of vectors $x$ and $y$ in $\mathcal{H}$, we have 
                \[\|x+y\|^2_\mathcal{H} + \|x-y\|^2_\mathcal{H} = 2 \cdot \|x\|_2^\mathcal{H} + 2\cdot \|y\|_\mathcal{H}^2.\]
                Using only characteristic functions of measurable subsets of $[0,1]$, prove that $L^1([0,1])$ is not a Hilbert space.
            \part[10]
                (Embedding of $\ell^p$ in $L^p[1,\infty)$.) Let $1\leq p<\infty$. For $a = (a_1, a_2, \dots)\in\ell^p$, define $f_a: [1,\infty) \to \mathbb{R}$ by 
                \[f_a(x)\coloneq \sum_{k=1}^\infty a_k\chi_{[k,k+1)}(x) \quad (x\in [1,\infty)).\]
                \begin{subparts}
                    \subpart[2]
                        Prove that $f_a\in L^p[1,\infty)$.
                    \subpart[2]
                        Prove that $\|f_a\|_p = \|a\|_p$.
                    \subpart[3]
                        Prove that the map $T: \ell^p \to L^p[1,\infty)$ given by $T(a)\coloneq f_a$ is linear and injective, but not surjective.
                    \subpart[3]
                        True or false? The range of $T$ is dense in $L^p[1,\infty)$.
                \end{subparts}
        \end{parts}
    \question[6]
        Let $f$ be a Lipschitz function on $[0,1]$ and let $g$ an absolutely continuous function on $[0,1]$. Prove that $f\circ g$ is absolutely continuous on $[0,1]$.
    \question[16]
        On a closed, bounded, nondegenerate interval $[a,b]$, consider a sequence $\{f_n\}$ of increasing, absolutely continuous functions on $[a,b]$. Assume:

        (i) $f_n(a)$ = 0 for all $n\in\mathbb{N}$;

        (ii) $f'_n\leq f'_{n+1}$ a.e. on $[a,b]$, for all $n\in\mathbb{N}$;

        (iii) The sequence $\{f_n(b)\}$ is bounded.

        Prove:
        \begin{parts}
            \part[4]
                There exists $f: [a,b] \to \mathbb{R}$ such that $\{f_n\}\to f$ pointwise on $[a,b]$;
            \part[6]
                $f$ is absolutely continuous on $[a,b]$; and
            \part[6]
                $\{f_n\}$ converges to $f$ uniformly on $[a,b]$.
        \end{parts}
    \question[14]
        \begin{parts}
           \part[7]
            For a nonempty subset $E$ of a metric space $(X, \rho)$ and a point $x\in X$, define the distance from $x$ to $E$ by 
            \[\text{dist}(x,E)\coloneq \inf\{\rho(x,y)\mid y\in E\}\].
            \begin{subparts}
                \subpart[4]
                    Show that $f$ is continuous on $X$.
                \subpart[3]
                    Show that $\overline{E}=f^{-1}(\{0\})$.
            \end{subparts} 
            \part[7]
            Now define $f: X \to \mathbb{R}$ by 
            \[f(x) \coloneq \text{dist}(x,E)\quad (x\in X).\]
            \begin{subparts}
               \subpart[4]
                    Assume that $F$ is closed, and $E$ is compact. Prove that 
                    \[E\cap F = \varnothing \iff \text{dist}(F, E) > 0.\]
                \subpart[3]
                    Prove that the previous statement is not true if $E$ and $F$ are merely closed. That is, find an example of two closed sets $E$ and $F$ for which 
                    dist$(F, E) = 0$ and $E\cap F = \varnothing$. (Hint: To produce an example in $\mathbb{R}$, by Heine-Borel both sets need to be closed and unbounded, e.g., consider $E\coloneq \mathbb{N}$.)
            \end{subparts}
        \end{parts}
\end{questions}

\end{document}